

%----------------------------------------------------------------------------------------
%	PACKAGES AND OTHER DOCUMENT CONFIGURATIONS
%----------------------------------------------------------------------------------------

\documentclass[a4paper, 11pt]{article} % Font size (can be 10pt, 11pt or 12pt) and paper size (remove a4paper for US letter paper)

\usepackage[protrusion=true,expansion=true]{microtype} % Better typography
\usepackage{graphicx} % Required for including pictures
\usepackage{wrapfig} % Allows in-line images
\usepackage{url}
\usepackage{amsmath}

%----------------------------------------------------------------------------------------%
%	                                       TITLE                                         %
%----------------------------------------------------------------------------------------%

\title{Detecting earthquake events} % Subtitle
\author{Lay Kuan Loh} % Author
\date{\today} % Date

%----------------------------------------------------------------------------------------

\begin{document}

\maketitle % Print the title section

% ---------------------------------------------------------------------------------------- %
% 	                                ABHIJIT GHOSH                                          %
% ---------------------------------------------------------------------------------------- %

\section{Abhijit Ghosh}

\subsection{Earthquake spectra and near-source attenuation in the Cascadia subduction zone}

\begin{enumerate}
	\item Found regular events for a short period of time using a combination of an automated detection scheme based on ratios of short-term to long-term average signal levels and visual verification. Signals for the events are temporally isolated, each have clear P and S saves, and spectral content/ recurrence is not considered.
	\item Low Frequency Earthquakes (LFE) are classified based on repeating occurence by using an algorithm that cross-correlates a template waveform with a moving window of a continuous data stream recorded at the same station. Repeats are noted well outside the time window of the study. 

	LFEs are identified as windows in which the correlation coefficients summed across 3-components of several stations exceeds a threshold value. 
	\item Find events for the entire period of time you want to study. If they are picked out by both method (1) and method (2), look for their repeats. Use cross-correlation analysis over time intervals to see how often these events have recurred. 
\end{enumerate}

\subsection{Tiny intraplate earthquakes triggered by nearby episodic tremor and slip in Cascadia}

\begin{enumerate}
	\item  
\end{enumerate}

% ---------------------------------------------------------------------------------------- %
% 	                                ABHIJIT GHOSH                                          %
% ---------------------------------------------------------------------------------------- %







% ---------------------------------------------------------------------------------------- %
% 	                                  GREG BEROZA                                          %
% ---------------------------------------------------------------------------------------- %

\section{Greg Beroza}

% ---------------------------------------------------------------------------------------- %
% 	                                  GREG BEROZA                                          %
% ---------------------------------------------------------------------------------------- %






% ---------------------------------------------------------------------------------------- %
% 	                               MICHAEL BRUDZINSKI                                      %
% ---------------------------------------------------------------------------------------- %

\section{Michael Brudzinski}



% ---------------------------------------------------------------------------------------- %
% 	                               MICHAEL BRUDZINSKI                                      %
% ---------------------------------------------------------------------------------------- %








%----------------------------------------------------------------------------------------

\end{document}