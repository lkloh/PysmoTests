%%%%%%%%%%%%%%%%%%%%%%%%%%%%%%%%%%%%%%%%%
% Thin Sectioned Essay
% LaTeX Template
% Version 1.0 (3/8/13)
%
% This template has been downloaded from:
% http://www.LaTeXTemplates.com
%
% Original Author:
% Nicolas Diaz (nsdiaz@uc.cl) with extensive modifications by:
% Vel (vel@latextemplates.com)
%
% License:
% CC BY-NC-SA 3.0 (http://creativecommons.org/licenses/by-nc-sa/3.0/)
%
%%%%%%%%%%%%%%%%%%%%%%%%%%%%%%%%%%%%%%%%%

%----------------------------------------------------------------------------------------
%	PACKAGES AND OTHER DOCUMENT CONFIGURATIONS
%----------------------------------------------------------------------------------------

\documentclass[a4paper, 11pt]{article} % Font size (can be 10pt, 11pt or 12pt) and paper size (remove a4paper for US letter paper)

\usepackage[protrusion=true,expansion=true]{microtype} % Better typography
\usepackage{graphicx} % Required for including pictures
\usepackage{wrapfig} % Allows in-line images
\usepackage{url}
\usepackage{mathpazo} % Use the Palatino font
\usepackage[T1]{fontenc} % Required for accented characters
\linespread{1.05} % Change line spacing here, Palatino benefits from a slight increase by default

\makeatletter
\renewcommand\@biblabel[1]{\textbf{#1.}} % Change the square brackets for each bibliography item from '[1]' to '1.'
\renewcommand{\@listI}{\itemsep=0pt} % Reduce the space between items in the itemize and enumerate environments and the bibliography

\renewcommand{\maketitle}{ % Customize the title - do not edit title and author name here, see the TITLE block below
\begin{flushright} % Right align
{\LARGE\@title} % Increase the font size of the title

\vspace{50pt} % Some vertical space between the title and author name

{\large\@author} % Author name
\\\@date % Date

\vspace{40pt} % Some vertical space between the author block and abstract
\end{flushright}
}

%----------------------------------------------------------------------------------------
%	TITLE
%----------------------------------------------------------------------------------------

\title{Seismogram Clustering Suggestions} % Subtitle
\author{Lay Kuan Loh} % Author
\date{\today} % Date

%----------------------------------------------------------------------------------------

\begin{document}

\maketitle % Print the title section

First Try

\begin{enumerate}
 	\item Given $n$ seismograms, called $s_1,\,s_2,\cdots,s_n$, each of which has $m$ datapoints in it, we want to correlate each interval along the seismograms with the rest of them, with intervals of, for instance, 60 seconds each. So, numbering the datapoints in $s_i$ by $\{t_1,\,t_2,\ldots,t_m\}$, we can choose groups of $\{t^i_1,\cdots,t^i_{60}\},\{t^i_{61},\cdots,t^i_{120}\},\ldots,\{t^i_{60\lfloor\frac{m}{60}\rfloor+ 1},\cdots,t^i_m\}$ in $s_i$. 

	\item To do clustering on the seismograms $s_1,\,s_2,\cdots,s_n$ for an event, it is best to only have one affinity matrix $A$ for the seismograms. You tweak the metric used to compute affinity between $s_i$ and $s_j$, not compute more affinity matrices. 

	\item Find the affinity between $s_i$ and $s_j$ by first separating them into groups $\{t^i_1,\cdots,t^i_{60}\},\{t^i_{61},\cdots,t^i_{120}\},\ldots,\{t^i_{60\lfloor\frac{m}{60}\rfloor+ 1},\cdots,t^i_m\}$ in $s_i$ and $\{t^j_1,\cdots,t^j_{60}\},\{t^j_{61},\cdots,t^j_{120}\},\ldots,\{t^j_{60\lfloor\frac{m}{60}\rfloor+ 1},\cdots,t^j_m\}$ respectively. To find the affinity, try finding the covariance between signals for the groups chosen. 

	\item Now find the local affinity matrix $B$ between these groups for $s_i$ and $s_j$. Choose the two portions $s_i^c$ and $s_j^c$ in $s_i$ and $s_j$ giving the highest affinity, and let $A(i,j) = \max(B)$. Remember to save where $s_i^c$ and $s_j^c$ start and end for each $s_i$ and $s_j$ pair. 

	\item Perform clustering using $A$. Try $k$-means first, then Spectral Clustering.
\end{enumerate}

Possible tweaks
\begin{itemize}
    \item Make the intervals a bit smaller, $<60$ seconds. Need to experiment with this a bit.
    \item Replace (4) above with doing clustering on matrix $B$. Find the cluster in $s_i$ and $s_j$ that has the highest affinity, and set that to be $A_{ij}$. So, double clustering. 
    \item In (3), instead of covariance, other possible metrics are correlation. Other metrics are here: \url{http://docs.scipy.org/doc/scipy/reference/spatial.distance.html}
\end{itemize}






%----------------------------------------------------------------------------------------

\end{document}