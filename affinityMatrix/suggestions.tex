

%----------------------------------------------------------------------------------------
%	PACKAGES AND OTHER DOCUMENT CONFIGURATIONS
%----------------------------------------------------------------------------------------

\documentclass[a4paper, 11pt]{article} % Font size (can be 10pt, 11pt or 12pt) and paper size (remove a4paper for US letter paper)

\usepackage[protrusion=true,expansion=true]{microtype} % Better typography
\usepackage{graphicx} % Required for including pictures
\usepackage{wrapfig} % Allows in-line images
\usepackage{url}
\usepackage{amsmath}

%----------------------------------------------------------------------------------------
%	TITLE
%----------------------------------------------------------------------------------------

\title{Seismogram Clustering Suggestions} % Subtitle
\author{Lay Kuan Loh} % Author
\date{\today} % Date

%----------------------------------------------------------------------------------------

\begin{document}

\maketitle % Print the title section

\section{First Try}

\begin{enumerate}
 	\item Given $n$ seismograms, called $s_1,\,s_2,\cdots,s_n$, each of which has $m$ datapoints in it, we want to correlate each interval along the seismograms with the rest of them, with intervals of, for instance, 60 seconds each. So, numbering the datapoints in $s_i$ by $\{t_1,\,t_2,\ldots,t_m\}$, we can choose groups of $\{t^i_1,\cdots,t^i_{60}\},\{t^i_{61},\cdots,t^i_{120}\},\ldots,\{t^i_{60\lfloor\frac{m}{60}\rfloor+ 1},\cdots,t^i_m\}$ in $s_i$. 

	\item To do clustering on the seismograms $s_1,\,s_2,\cdots,s_n$ for an event, it is best to only have one affinity matrix $A$ for the seismograms. You tweak the metric used to compute affinity between $s_i$ and $s_j$, not compute more affinity matrices. 

	\item Find the affinity between $s_i$ and $s_j$ by first separating them into groups $\{t^i_1,\cdots,t^i_{60}\},\{t^i_{61},\cdots,t^i_{120}\},\ldots,\{t^i_{60\lfloor\frac{m}{60}\rfloor+ 1},\cdots,t^i_m\}$ in $s_i$ and $\{t^j_1,\cdots,t^j_{60}\},\{t^j_{61},\cdots,t^j_{120}\},\ldots,\{t^j_{60\lfloor\frac{m}{60}\rfloor+ 1},\cdots,t^j_m\}$ respectively. To find the affinity, try finding the covariance between signals for the groups chosen. 

	\item Now find the local affinity matrix $B$ between these groups for $s_i$ and $s_j$. Choose the two portions $s_i^c$ and $s_j^c$ in $s_i$ and $s_j$ giving the highest affinity, and let $A(i,j) = \max(B)$. Remember to save where $s_i^c$ and $s_j^c$ start and end for each $s_i$ and $s_j$ pair. 

	\item Perform clustering using $A$. Try $k$-means first, then Spectral Clustering.
\end{enumerate}

\section{Possible tweaks}

\begin{itemize}
    \item Make the intervals a bit smaller, $<60$ seconds. Need to experiment with this a bit.

    \item Replace (4) above with doing clustering on matrix $B$. Find the cluster in $s_i$ and $s_j$ that has the highest affinity, and set that to be $A_{ij}$. So, double clustering. 

    \item In (3), instead of covariance, other possible metrics are correlation. Other metrics are here: \url{http://docs.scipy.org/doc/scipy/reference/spatial.distance.html}

	\item Alternatively, instead of doing all that, for each pair of seismograms $s_i$ and $s_j$, transform the time signal data to frequency signal data, and do clustering on that instead. Filter and get write of peaks below a certain minimum amplitude. Find the covariance between the transformed seismograms.
\end{itemize}






%----------------------------------------------------------------------------------------

\end{document}